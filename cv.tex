%% start of file `template.tex'.
%% Copyright 2006-2013 Xavier Danaux (xdanaux@gmail.com).
%
% This work may be distributed and/or modified under the
% conditions of the LaTeX Project Public License version 1.3c,
% available at http://www.latex-project.org/lppl/.


\documentclass[11pt,letterpaper,sans]{moderncv}        % possible options include font size ('10pt', '11pt' and '12pt'), paper size ('a4paper', 'letterpaper', 'a5paper', 'legalpaper', 'executivepaper' and 'landscape') and font family ('sans' and 'roman')
\usepackage{xspace}
\newcommand\eat[1]{}
%\newcommand{\ie}{{\it i.e.,}\xspace}

% moderncv themes
\moderncvstyle{classic}                             % style options are 'casual' (default), 'classic', 'oldstyle' and 'banking'
\moderncvcolor{blue}                               % color options 'blue' (default), 'orange', 'green', 'red', 'purple', 'grey' and 'black'
%\renewcommand{\familydefault}{\sfdefault}         % to set the default font; use '\sfdefault' for the default sans serif font, '\rmdefault' for the default roman one, or any tex font name
%\nopagenumbers{}                                  % uncomment to suppress automatic page numbering for CVs longer than one page

% character encoding
%\usepackage[utf8]{inputenc}                       % if you are not using xelatex ou lualatex, replace by the encoding you are using
%\usepackage{CJKutf8}                              % if you need to use CJK to typeset your resume in Chinese, Japanese or Korean

% adjust the page margins
\usepackage[scale=0.83]{geometry}
\setlength{\hintscolumnwidth}{3.2cm}                % if you want to change the width of the column with the dates
%\setlength{\makecvtitlenamewidth}{10cm}           % for the 'classic' style, if you want to force the width allocated to your name and avoid line breaks. be careful though, the length is normally calculated to avoid any overlap with your personal info; use this at your own typographical risks...

\newcommand\AH[1]{\textcolor{red}{#1}}
\newcommand{\myname}[1]{\textbf{#1}}


% personal data
\name{Ahmed}{El-Hassany}
\title{Curriculum Vitae}                               % optional, remove / comment the line if not wanted
%\address{1639 Oxford St, APT 3}{Berkeley, CA 94709}   % optional, remove / comment the line if not wanted; the "postcode city" and "country" arguments can be omitted or provided empty
%\phone[mobile]{+1~(302)~379~4791}                   % optional, remove / comment the line if not wanted; the optional "type" of the phone can be "mobile" (default), "fixed" or "fax"
%\phone[fixed]{+2~(345)~678~901}
%\phone[fax]{+3~(456)~789~012}
\email{eahmed@ethz.ch}                               % optional, remove / comment the line if not wanted
\renewcommand*\httplink[2][]{{\urlstyle{sf}\expandafter\href#2}}
\homepage{{https://hassany.ps}{https://hassany.ps}}                         % optional, remove / comment the line if not wanted
%\social[linkedin]{john.doe}                        % optional, remove / comment the line if not wanted
%\social[twitter]{jdoe}                             % optional, remove / comment the line if not wanted
\social[github]{ahassany}                              % optional, remove / comment the line if not wanted
%\extrainfo{additional information}                 % optional, remove / comment the line if not wanted
%\photo[64pt][0.4pt]{picture}                       % optional, remove / comment the line if not wanted; '64pt' is the height the picture must be resized to, 0.4pt is the thickness of the frame around it (put it to 0pt for no frame) and 'picture' is the name of the picture file
%\quote{Some quote}                                 % optional, remove / comment the line if not wanted

% to show numerical labels in the bibliography (default is to show no labels); only useful if you make citations in your resume
%\makeatletter
%\renewcommand*{\bibliographyitemlabel}{\@biblabel{\arabic{enumiv}}}
\makeatother
\renewcommand*{\bibliographyitemlabel}{[\arabic{enumiv}]}% CONSIDER REPLACING THE ABOVE BY THIS

% bibliography with mutiple entries
\usepackage{multibib}
\newcites{conference,workshop,demos,trs}{{Conferences},{Workshops},{Demos},{Technical Reports}}
%----------------------------------------------------------------------------------
%            content
%----------------------------------------------------------------------------------
\begin{document}
%\begin{CJK*}{UTF8}{gbsn}                          % to typeset your resume in Chinese using CJK
%-----       resume       ---------------------------------------------------------
\makecvtitle
\section{Research Interests}
Networks, Software-Defined Networks, Systems, Distributed Systems, Program Synthesis. %, High Performance Computing. Networks Programmability

\section{Education}
\cventry{2015--present}{PhD student}{ETH Z\"urich}{Switzerland}{}{\emph{Advised by:} Prof. Laurent Vanbever}  %
%\cventry{2011--2015}{PhD student (not finished)}{Indiana University}{Bloomington, IN}{}{GPA 3.80/4.0}  % arguments 3 to 6 can be left empty
\cventry{2009--2011}{M.S. Computer Science}{University of Delaware}{Newark, DE}{}{}%{GPA 3.58/4.0}
\cventry{2003--2008}{B.Sc. Computer Engineering}{Islamic University of Gaza}{Gaza, Palestine}{}{}%{GPA 82.5/100}


\section{Awards}
\cvlistitem{Fulbright Foreign Student Scholarship for Master's program, 2009-2011.}
\cvlistitem{2\textsuperscript{nd} place winner team at IEEE/ACM Supercomputer Conference 2009 High Performance Computing Contest.}
\cvlistitem{Google Summer of Code, 2008.}
\cvlistitem{Palestinian Prime Minister's special award for undergraduate achievements, 2008.}


%\section{Recent Research Projects}

%\cvlistitem {\textbf{SDNv2}: We designed new a network architecture that changes the innovation model in the network.  The main architectural changes we propose are (1) using software based at the network edge (2) including middleboxes as a fundamental component in the architecture, as opposed to the current architectures that ignore their existence (3) extending network virtualization to higher-level services to open the network for third-party services and network function virtualization (NFV).}

%The Software-Defined Networking (SDN) paradigm, as canonically exemplified by the universal adoption of OpenFlow in hardware switches controlled by logically centralized controllers, was based on several implicit assumptions about networks. In this project we revisit those assumptions and find them wanting. We then propose a revised approach, which we call SDNv2 that involves (i) extending network virtualization to higher-level services, and (ii) implementing these and other functionalities on a more modular and flexible SDN architecture.}

%\cvlistitem {\textbf{STS}: Industry providers for Software-Defined Networks (SDN) controller built large testing infrastructures to test their controllers. These testing infrastructures stress the controllers for long period of time, to simulate real network operating scenarios, until it find a bug in the controller. We have been told by some of these vendors that the input event traces that leads to that bug is large and it's very hard to troubleshoot a bug. Our work focuses on how to minimize the input event traces to make it reasonable enough for a developer to troubleshoot the discovered bug. We apply our technique to five open source SDN control platforms --Foodlight, NOX, POX, Pyretic, ONOS-- and illustrate how the minimal causal sequences our system found aided the troubleshooting process.\\ ONOS quality assurance team uses parts of \emph{STS} for integration and continuous operation test suits.}

%Software bugs are inevitable in software-defined networking control software, and troubleshooting is a tedious, time-consuming task. We worked on developing techniques for automatically identifying a minimal sequence of inputs responsible for triggering a given bug in control software, without making assumptions about the language or instrumentation of the software under test. We apply our technique to five open source SDN control platforms --Foodlight, NOX, POX, Pyretic, ONOS-- and illustrate how the minimal causal sequences our system found aided the troubleshooting process.\\
% ONOS quality assurance team uses parts of \emph{STS} for integration and continuous operation test suits.}

%\cvlistitem { \textbf{GEMINI}: Develop and deploy instrumentation and measurement framework, capable of supporting the needs of both GENI experimenters and GENI infrastructure operators. It uses the perfSONAR system as its basis, and includes additional capabilities being developed by key GENI I\&M projects. It collects and manages both substrate metrics as well as slice-specific measurements. It introduces a GENI Global I\&M Information Service so that GEMINI components can locate one another, locate measurement sources and data, and correlate measurement data to physical and virtual network topology. It includes access control for instrumentation infrastructure, measurements, and measurement data based on GENI policy and authorization mechanisms.\\ \emph{GEMINI} tools are currently used to instrument and measure GENI experiments: \url{http://groups.geni.net/geni/wiki/GEMINI/Tutorial}.\\ \emph{GEMINI} APIs and data models are used by the GENI network operators to to expose physcial infrastructure information to the users \url{http://groups.geni.net/geni/wiki/OperationalMonitoring/DataSchema}. }

\section{Publications}
\nociteconference{*}
\bibliographystyleconference{IEEEtran}
\bibliographyconference{conference}                   % 'publications' is the name of a BibTeX file

\nociteworkshop{*}
\bibliographystyleworkshop{IEEEtran}
\bibliographyworkshop{workshop}                   % 'publications' is the name of a BibTeX file


\nocitetrs{*}
\bibliographystyletrs{plainyr-rev}
\bibliographytrs{tr}                   % 'publications' is the name of a BibTeX file

\nocitedemos{*}
\bibliographystyledemos{IEEEtran}
\bibliographydemos{demos}                   % 'publications' is the name of a BibTeX file



\section{Professional Experience}

% -------------- ETH--------------------
\cventry{June '15 -- Present}{Research Assistant}{ETH Z\"urich}{Switzerland}{}{\begin{itemize}
  %\item Network Programmability.
  \item SDN Verification \url{http://sdnracer.ethz.ch}. Published in \cite{bigbug, sdnracer2, sdnracer}.
  \item Network-wide Configuration Synthesis \url{http://synet.ethz.ch}. Published in \cite{synet-cav, synet-arxiv}.
\end{itemize}}


% -------------- IU 2nd --------------------
\cventry{Spring '15}{Research Associate}{Indiana University}{Bloomington, IN}{}{\begin{itemize}
\item Measured garbage collector and data serialization overhead for unstructured data.
\item Worked on developing efficient methods for representing data in Haskell's runtime system. Published in \cite{cnf}.
\end{itemize}}

% -------------- ICSI --------------------
\cventry{July '13 -- Nov. '14}{Research Scientist}{International Computer Science Institute}{Berkeley, CA}{}{\begin{itemize}
\item Worked designing and building next generation SDN architecture (SDNv2).
\item Work on integrating STS project with ONOS; an open source SDN controller \url{http://onosproject.org/} and our work is published in \cite{scott2014trooubleshooting}.
\item Help ONOS QA team adopt parts of STS in their testing infrastructure.
\end{itemize}}

% -------------- ESnet --------------------
\cventry{Summer '13}{Summer Student}{Lawrence Berkeley National Laboratory}{Berkeley, CA}{}{\begin{itemize}
\item Worked with the Energy Sciences Network (ESnet) team.
\item Designed and developed scalable multi-domain Topology Service for dynamic multi-domain network circuits' setup.
\item \emph{Technology used:} Java, Python, Tornado, MongoDB.
\end{itemize}}

% -------------- IU --------------------
\cventry{Aug. '11 -- July '13}{Research Associate}{Indiana University}{Bloomington, IN}{}{\begin{itemize}
\item Designed and developed a RESTful Unified Network Information Services (UNIS) to represent topologies for large-scale multi-domain networks.
\item Designed and developed instrumentation and monitoring APIs and tools for GENI experiments and physical infrastructure.
\item \emph{Technology used:} Java, Python, Tornado, MongoDB.
\end{itemize}}

% --------------- SC -- maybe ignore it --------

% -------------- LBL --------------------
\cventry{Summer '10}{Summer Student}{Lawrence Berkeley National Laboratory}{Berkeley, CA}{}{\begin{itemize}
\item Worked at the Center for Enabling Distributed Petascale Computing (CEDPS) project.
\item Worked on designing and building system to collect, represent and analyze large scale  monitoring data for high-speed data transfers in DOE networks.
\item \emph{Technology used:} Java, Python, Tornado, MongoDB.
\end{itemize}}


% -------------- MoG --------------------
\cventry{Feb. '09 -- July '09}{Software Engineer}{Municipality of Gaza}{Gaza, Palestine}{}{\begin{itemize}
\item Designed a system to extract information, categorize and archive old, from late 1800s, building permits.
\item Designed new business process and the required software for issuing building permits in Gaza.
\item Lead a team of 50 civil engineers and 20 data entry persons to implement the project.
\item \emph{Technology used:} Oracle RDBMS, Visual Basic .NET, Delphi.
\end{itemize}}


% -------------- PNINA --------------------
\cventry{Dec. '08 -- Mar. '09}{Independent Consultant}{Palestinian National Internet Naming Authority (PNINA)}{Gaza, Palestine}{}{\begin{itemize}
\item Software quality assurance of in house built system for registering managing all .ps domain names; PNINA is the country code top-level domain (ccTLD) for Palestine.
\item Consulting on deployment and integration for new domain registration system.
\item \emph{Technology used:} PHP, Perl, PostgreSQL, MySQL, BIND, Apache.
\end{itemize}}

% -------------- Google Summer of Code --------------------
\cventry{Summer '08}{Student Developer}{Google Inc. \& Internet2}{}{Google Summer of Code 2008}{\begin{itemize}
\item This project is sponsored by Google and administrated by Internet2.
\item Designed and developed open source web based configuration tools for perfSONAR-PS.
\item \emph{Technology used:} Perl, JavaScript, Apache, HTML, CSS.
\end{itemize}}

% -------------- AfkarIT --------------------
\cventry{Sep. '07 -- Mar. '08}{Software Engineer}{AfkarIT}{}{Gaza, Palestine}{\begin{itemize}
\item Designed and developed operating system level virtual machines monitoring system for VMWare ESX Infrastructure.
\item \emph{Technology used:} Python, C\#.
\end{itemize}}


% -------------- Ard El-Insan --------------------
\cventry{Sep. '04 -- Oct. '05}{Contractor Software Engineer}{Ard El-Insan}{}{Gaza, Palestine}{\begin{itemize}
\item Designed and developed patients follow-up management system in four clinics for a project sponsored by European Commission's Humanitarian Aid Office (ECHO).
\item \emph{Technology used:} Visual Basic. NET, Microsoft SQL Server.
\end{itemize}}



\section{Activities and Services}
\cvlistitem{Reviewer: IEEE/ACM Transactions on Networking.}
\cvlistitem{Attended Par Lab Boot Camp -- short course on parallel programming, UC Berkeley 2010.}
\cvlistitem{SCinet student volunteer, Supercomputing Conference 2010, 2011, 2012.}
\cvlistitem{President of Graduate Students Association for Computer and Information Science Department at University of Delaware, (Fall 2010 -- Spring 2011)}
\cvlistitem{Attended From Lab to Market, Fulbright Enrichment Seminar, Austin TX, June 1\textsuperscript{st}-5\textsuperscript{th} 2009.}

\end{document}


%% end of file `template.tex'.
