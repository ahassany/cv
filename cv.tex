%% start of file `template.tex'.
%% Copyright 2006-2013 Xavier Danaux (xdanaux@gmail.com).
%
% This work may be distributed and/or modified under the
% conditions of the LaTeX Project Public License version 1.3c,
% available at http://www.latex-project.org/lppl/.


\documentclass[11pt,a4paper,sans]{moderncv}        % possible options include font size ('10pt', '11pt' and '12pt'), paper size ('a4paper', 'letterpaper', 'a5paper', 'legalpaper', 'executivepaper' and 'landscape') and font family ('sans' and 'roman')
\usepackage{xspace}

\usepackage[T1]{fontenc}
\usepackage{enumitem}

\setlist[itemize]{itemsep=0.3em, align=left, label=\itemlabel}


\newcommand\eat[1]{}
%\newcommand{\ie}{{\it i.e.,}\xspace}

% moderncv themes
\moderncvstyle{classic}                             % style options are 'casual' (default), 'classic', 'oldstyle' and 'banking'
\moderncvcolor{grey}                               % color options 'blue' (default), 'orange', 'green', 'red', 'purple', 'grey' and 'black'
%\renewcommand{\familydefault}{\sfdefault}         % to set the default font; use '\sfdefault' for the default sans serif font, '\rmdefault' for the default roman one, or any tex font name
%\nopagenumbers{}                                  % uncomment to suppress automatic page numbering for CVs longer than one page

% character encoding
%\usepackage[utf8]{inputenc}                       % if you are not using xelatex ou lualatex, replace by the encoding you are using
%\usepackage{CJKutf8}                              % if you need to use CJK to typeset your resume in Chinese, Japanese or Korean

% adjust the page margins
\usepackage[scale=0.80]{geometry}
\setlength{\hintscolumnwidth}{3.2cm}                % if you want to change the width of the column with the dates
%\setlength{\makecvtitlenamewidth}{10cm}           % for the 'classic' style, if you want to force the width allocated to your name and avoid line breaks. be careful though, the length is normally calculated to avoid any overlap with your personal info; use this at your own typographical risks...

\newcommand\AH[1]{\textcolor{red}{#1}}
\newcommand{\myname}[1]{\textbf{#1}}
\def\industry{}
%\def\pic{}


% personal data
\name{Ahmed}{El-Hassany}
%\title{Curriculum Vitae}                               % optional, remove / comment the line if not wanted
%\address{1639 Oxford St, APT 3}{Berkeley, CA 94709}   % optional, remove / comment the line if not wanted; the "postcode city" and "country" arguments can be omitted or provided empty
\address{Kingenstr. 40, 8005 Z\"urich}   % optional, remove / comment the line if not wanted; the "postcode city" and "country" arguments can be omitted or provided empty
\phone[mobile]{+41~76~730~87~96}                   % optional, remove / comment the line if not wanted; the optional "type" of the phone can be "mobile" (default), "fixed" or "fax"
\email{a.hassany@gmail.com}                               % optional, remove / comment the line if not wanted
%\renewcommand*\httplink[2][]{{\urlstyle{sf}\expandafter\href{#2}}
\homepage{hassany.ps}                         % optional, remove / comment the line if not wanted
%\social[linkedin]{john.doe}                        % optional, remove / comment the line if not wanted
%\social[twitter]{jdoe}                             % optional, remove / comment the line if not wanted
\social[github]{ahassany}                              % optional, remove / comment the line if not wanted
%\extrainfo{additional information}                 % optional, remove / comment the line if not wanted
\nationality{Palestinian}
\birthday{Born April 1\textsuperscript{st}, 1986}
\ifdefined\pic
	\photo[64pt][0.4pt]{pic.jpeg}                       % optional, remove / comment the line if not wanted; '64pt' is the height the picture must be resized to, 0.4pt is the thickness of the frame around it (put it to 0pt for no frame) and 'picture' is the name of the picture file
\fi
%\quote{Some quote}                                 % optional, remove / comment the line if not wanted

% to show numerical labels in the bibliography (default is to show no labels); only useful if you make citations in your resume
%\makeatletter
%\renewcommand*{\bibliographyitemlabel}{\@biblabel{\arabic{enumiv}}}
\makeatother
\renewcommand*{\bibliographyitemlabel}{[\arabic{enumiv}]}% CONSIDER REPLACING THE ABOVE BY THIS

% bibliography with mutiple entries
\usepackage{multibib}
\newcites{conference,workshop,demos,trs}{{Conferences},{Workshops},{Demos},{Technical Reports}}

%----------------------------------------------------------------------------------
%            content
%----------------------------------------------------------------------------------
\begin{document}
%-----       resume       ---------------------------------------------------------
\makecvtitle

\ifdefined\industry
%\section{Interests}
%\cvline{}{Building systems that solve challenging business requirements.} \medskip
\else
\section{Research Interests}
\cvline{}{Networks, Software-Defined Networks, Systems, Distributed Systems, Program Synthesis.} \medskip
\fi

\section{Education}
\cventry{2015--Summer 2019}{PhD student}{ETH Z\"urich}{Switzerland}{}{\emph{Advised by:} Prof. Laurent Vanbever}  %
%\cventry{2011--2015}{PhD student (not finished)}{Indiana University}{Bloomington, IN}{}{GPA 3.80/4.0}  % arguments 3 to 6 can be left empty
\cventry{2009--2011}{M.S. Computer Science}{University of Delaware}{Newark, DE, USA}{}{}%{GPA 3.58/4.0}
\cventry{2003--2008}{B.Sc. Computer Engineering}{Islamic University of Gaza}{Gaza, Palestine}{}{}%{GPA 82.5/100}

\medskip
\ifdefined\industry
	\section{Professional Experience}


% -------------- ETH--------------------
\cventry{June '15 -- Present}{Research Assistant}{ETH Z\"urich}{Switzerland}{}{\begin{itemize}
  \item Developed systems to automatically synthesize router configurations from high-level requirements.  \httplink[http://netcomplete.ethz.ch]{http://netcomplete.ethz.ch}, \httplink[http://synet.ethz.ch]{http://synet.ethz.ch}. Published in \cite{netcomplete, synet-cav, synet-arxiv}.
  \item Developed a system to detect concurrency violations in production-grade controllers of Software-Defined Networks (SDN).  \httplink[http://sdnracer.ethz.ch]{http://sdnracer.ethz.ch}. Published in \cite{bigbug, sdnracer2, sdnracer}.
  \item \emph{Technologies used:} Cisco IOS, BGP, OSPF, Python, SDN, Z3 SMT Solver, Git.
\end{itemize}}


% -------------- Facebook--------------------
\cventry{June -- Sept. '18}{Software Engineer Intern}{Facebook}{Menlo Park, CA, USA}{}{\begin{itemize}
  \item Started a practical network verification initiative.
  \item Built a prototype system to verify the correctness of the forwarding state of Facebook's network.
  \item \emph{Technologies used:} Python, Thrift, Mercurial.
\end{itemize}}


% -------------- IU 2nd --------------------
\cventry{Jan -- May '15}{Research Associate}{Indiana University}{Bloomington, IN, USA}{}{\begin{itemize}
\item Measured garbage collector and data serialization overhead for unstructured data.
\item Developed efficient method for representing unstructured data in Haskell's runtime system. Published in \cite{cnf}.
\item \emph{Technologies used:} Java, Git.
\end{itemize}}

% -------------- ICSI --------------------
\cventry{July '13 -- Nov. '14}{Research Scientist}{International Computer Science Institute (ICSI)}{Berkeley, CA, USA}{}{\begin{itemize}
\item Worked in Prof. Scott Shenker's group on designing and building next-generation SDN architecture (SDNv2).
\item Integrated a research system for quality assurance to run in a production environement with ONOS; a carrier-grade SDN open-source network operating system. \httplink[http://onosproject.org/]{http://onosproject.org}. Published in \cite{scott2014trooubleshooting}.
%\item Helped ONOS Quality Assurance team adopt parts of STS in their testing infrastructure.
\item \emph{Technologies used:} Python, Java, Continuous Integration QA, Git.
\end{itemize}}

% -------------- ESnet --------------------
\cventry{May -- July '13}{Summer Student}{Lawrence Berkeley National Lab.}{Berkeley, CA, USA}{}{\begin{itemize}
\item Worked with the Energy Sciences Network (ESnet) team.
\item Developed a scalable multi-domain Network Topology Service for dynamic multi-domain network circuits' setup.
\item \emph{Technologies used:} Java, Jetty, Atmoshper Framework, REST API design, MongoDB.
\end{itemize}}

% -------------- IU --------------------
\cventry{Aug. '11 -- July '13}{Research Associate}{Indiana University}{Bloomington, IN, USA}{}{\begin{itemize}
\item Developed a RESTful Unified Network Information Services (UNIS) to represent topologies for large-scale multi-domain networks. Published in \cite{ahassany2013unis}.
\item Developed instrumentation and monitoring APIs and tools for GENI \httplink[https://www.geni.net]{https://www.geni.net} experiments and physical infrastructure; \httplink[https://github.com/periscope-ps/unis]{github.com/periscope-ps/unis}.
\item \emph{Technologies used:} Java, Python, Tornado, REST API design, MongoDB, Django, Git.
\end{itemize}}

% --------------- SC -- maybe ignore it --------

% -------------- LBL --------------------
\cventry{August '10}{Summer Student}{Lawrence Berkeley National Lab.}{Berkeley, CA, USA}{}{\begin{itemize}
\item Worked at the Center for Enabling Distributed Petascale Computing (CEDPS) project.
\item Developed a system to collect, represent and analyze large scale  monitoring data for high-speed data transfers in DOE networks.
\item \emph{Technologies used:} Python, Tornado, JavaScript, HTML, JQuery, MongoDB, Django.
\end{itemize}}


% -------------- MoG --------------------
\cventry{Feb. '09 -- July '09}{Software Engineer}{Municipality of Gaza}{Gaza, Palestine}{}{\begin{itemize}
\item Designed a system to extract information, categorize, and archive old building permits from the late 1800s.
\item Designed a new business process and the required software for issuing new building permits in Gaza city.
\item Lead a team of 50 civil engineers and 20 data entry persons to implement the project.
\item \emph{Technologies used:} Oracle DB, Visual Basic~.NET, Delphi.
\end{itemize}}


% -------------- PNINA --------------------
\cventry{Dec. '08 -- Mar. '09}{Independent Consultant}{Palestinian National Internet Naming Authority (PNINA)}{Gaza, Palestine}{}{\begin{itemize}
\item Software quality assurance of in-house built system for registering and managing all \emph{.ps} domain names; PNINA is the country code top-level domain (ccTLD) for Palestine.
\item Consulting on deployment and integration for new domain registration system.
\item \emph{Technologies used:} PHP, Perl, PostgreSQL, MySQL, BIND, Apache.
\end{itemize}}

% -------------- Google Summer of Code --------------------
\cventry{June -- Sept '08}{Student Developer}{Google Inc. \& Internet2}{}{Google Summer of Code 2008}{\begin{itemize}
\item This project is sponsored by Google and administrated by Internet2.
\item Designed and developed open source web based configuration tools for perfSONAR-PS.
\item \emph{Technologies used:} Perl, JavaScript, Apache, HTML, CSS, SVN.
\end{itemize}}

% -------------- AfkarIT --------------------
\cventry{Sep. '07 -- Mar. '08}{Software Engineer}{AfkarIT}{}{Gaza, Palestine}{\begin{itemize}
\item Developed operating system level virtual machines monitoring system for VMWare ESX Infrastructure.
\item \emph{Technologies used:} Python, C\#.
\end{itemize}}


% -------------- Ard El-Insan --------------------
\cventry{Sep. '04 -- Oct. '05}{Contractor Software Engineer}{Ard El-Insan}{}{Gaza, Palestine}{\begin{itemize}
\item Developed patients follow-up management system in four clinics for a project sponsored by European Commission's Humanitarian Aid Office (ECHO).
\item \emph{Technologies used:} Visual Basic \mbox{.NET}, Microsoft SQL Server.
\end{itemize}}


	\medskip
	\section{Awards}
\cvlistitem{Fulbright Foreign Student Scholarship for Master's program, 2009-2011.}
\cvlistitem{2\textsuperscript{nd} place winner team at IEEE/ACM Supercomputer Conference 2009 (SC '09) High Performance Computing Contest.}
\cvlistitem{Google Summer of Code, 2008.}
\cvlistitem{Palestinian Prime Minister's special award for undergraduate achievements, 2008.}
	\medskip
	\section{Activities and Services}
\cvlistitem{Reviewer: IEEE/ACM Transactions on Networking.}
\cvlistitem{Attended Par Lab Boot Camp -- short course on parallel programming, UC Berkeley 2010.}
\cvlistitem{SCinet student volunteer, Supercomputing Conference 2010, 2011, 2012.}
\cvlistitem{President of Graduate Students Association for Computer and Information Science Department at University of Delaware, (Fall 2010 -- Spring 2011)}
\cvlistitem{Attended From Lab to Market, Fulbright Enrichment Seminar, Austin TX, June 1\textsuperscript{st}-5\textsuperscript{th} 2009.}
	\medskip
	\section{Presentations and invited talks}
 
\cventry{}{ Network-wide Configuration Synthesis}{}{}{}{\begin{itemize}
\item Computer Aided Verification (CAV'17), \href{https://youtu.be/dUbFWtHLTCI}{https://youtu.be/dUbFWtHLTCI}, July 2017, Heidelberg Germany.
\item Workshop on Network Verification \href{http://network-programming.org/wnv/}{http://network-programming.org/wnv/}. April 2017. Palo Alto, CA, USA.
\end{itemize}
}
 
\cventry{}{BigBug: Practical Concurrency Analysis for SDN }{}{}{}{\begin{itemize}
\item ACM SIGPLAN conference on Programming Language Design and Implementation (PLDI'16), talk available at \href{https://youtu.be/QzYBOc3G9FU}{https://youtu.be/QzYBOc3G9FU}. June 2016. Santa Barbara, CA, USA.
\end{itemize}
}

\cventry{}{SDNRacer: Concurrency Analysis for Software-Defined Networks}{}{}{}{\begin{itemize}
\item ACM Symposium on SDN Research (SOSR). April 2017. Santa Clara, CA, USA.
\end{itemize}
}

\cventry{}{UNIS: Design and Implementation of a Unified Network Information Service}{}{}{}{\begin{itemize}
\item IEEE Conference on Services Computing (IEEE SCC). July 2013. Santa Clara, CA USA.
\end{itemize}
}
	\medskip
	%\section{Recent Research Projects}

\cvlistitem {\textbf{SDNv2}: We designed new a network architecture that changes the innovation model in the network.  The main architectural changes we propose are (1) using software based at the network edge (2) including middleboxes as a fundamental component in the architecture, as opposed to the current architectures that ignore their existence (3) extending network virtualization to higher-level services to open the network for third-party services and network function virtualization (NFV).}

%The Software-Defined Networking (SDN) paradigm, as canonically exemplified by the universal adoption of OpenFlow in hardware switches controlled by logically centralized controllers, was based on several implicit assumptions about networks. In this project we revisit those assumptions and find them wanting. We then propose a revised approach, which we call SDNv2 that involves (i) extending network virtualization to higher-level services, and (ii) implementing these and other functionalities on a more modular and flexible SDN architecture.}

\cvlistitem {\textbf{STS}: Industry providers for Software-Defined Networks (SDN) controller built large testing infrastructures to test their controllers. These testing infrastructures stress the controllers for long period of time, to simulate real network operating scenarios, until it find a bug in the controller. We have been told by some of these vendors that the input event traces that leads to that bug is large and it's very hard to troubleshoot a bug. Our work focuses on how to minimize the input event traces to make it reasonable enough for a developer to troubleshoot the discovered bug. We apply our technique to five open source SDN control platforms --Foodlight, NOX, POX, Pyretic, ONOS-- and illustrate how the minimal causal sequences our system found aided the troubleshooting process.\\ ONOS quality assurance team uses parts of \emph{STS} for integration and continuous operation test suits.}

%Software bugs are inevitable in software-defined networking control software, and troubleshooting is a tedious, time-consuming task. We worked on developing techniques for automatically identifying a minimal sequence of inputs responsible for triggering a given bug in control software, without making assumptions about the language or instrumentation of the software under test. We apply our technique to five open source SDN control platforms --Foodlight, NOX, POX, Pyretic, ONOS-- and illustrate how the minimal causal sequences our system found aided the troubleshooting process.\\
% ONOS quality assurance team uses parts of \emph{STS} for integration and continuous operation test suits.}

\cvlistitem { \textbf{GEMINI}: Develop and deploy instrumentation and measurement framework, apable of supporting the needs of both GENI experimenters and GENI infrastructure operators. It uses the perfSONAR system as its basis, and includes additional capabilities being developed by key GENI I\&M projects. It collects and manages both substrate metrics as well as slice-specific measurements. It introduces a GENI Global I\&M Information Service so that GEMINI components can locate one another, locate measurement sources and data, and correlate measurement data to physical and virtual network topology. It includes access control for instrumentation infrastructure, measurements, and measurement data based on GENI policy and authorization mechanisms.\\ \emph{GEMINI} tools are currently used to instrument and measure GENI experiments: \href{http://groups.geni.net/geni/wiki/GEMINI/Tutorial}{http://groups.geni.net/geni/wiki/GEMINI/Tutorial}.\\ \emph{GEMINI} APIs and data models are used by the GENI network operators to to expose physcial infrastructure information to the users \href{http://groups.geni.net/geni/wiki/OperationalMonitoring/DataSchema}{http://groups.geni.net/geni/wiki/OperationalMonitoring/DataSchema}. }
  \section{Supervised Students}
% Swisscom
\cvlistitem{Germain Zouein, Master Thesis, ``Graph-based Management and Visualization of Network Infrastructure,'' Swisscom Digital Lab/EPFL, 2021.}
\cvlistitem{Andrej Janchevski, Master Thesis, ``Graph Embedding Methods for Graph Completion,'' Swisscom Digital Lab/EPFL, 2021.}
\cvlistitem{Vincent Coriou, Master Thesis, ``Knowledge Graph Reasoning for Infrastructure Elements,'' Swisscom Digital Lab/EPFL, 2021.}
\cvlistitem{Nicolas Zimmermann, Master Thesis, ``Link Prediction on Knowledge Graphs for Infrastructure Elements,'' Swisscom Digital Lab/EPFL, 2021.}
% ETH
\cvlistitem{Alexander Hedges, Semester Thesis, ``Grigori: Continuous Integration Testing of Synthesize Router Configurations,'' ETH Z\"urich, 2018.}
\cvlistitem{Christelle Gloor, Semester Thesis, ``Chronos: Finding the Configurations Recipe for Fast Convergence,'' ETH Z\"urich, 2017.}
\cvlistitem{Roman May, Master Thesis, ``BigBug: Practical Concurrenty Analaysis for SDN,'' ETH Z\"urich, 2016.}

	\medskip
  \newpage
	\section{Publications}
\medskip
\nociteconference{*}
\bibliographystyleconference{IEEEtran}
\bibliographyconference{IEEEabrv,conference}                   % 'publications' is the name of a BibTeX file

\medskip
\medskip
\nociteworkshop{*}
\bibliographystyleworkshop{IEEEtran}
\bibliographyworkshop{workshop}                   % 'publications' is the name of a BibTeX file

\medskip
\medskip
\nocitetrs{*}
\bibliographystyletrs{IEEEtran}
\bibliographytrs{tr}                   % 'publications' is the name of a BibTeX file
\medskip
\nocitedemos{*}
\bibliographystyledemos{IEEEtran}
\bibliographydemos{demos}                   % 'publications' is the name of a BibTeX file
	
	%\medskip
	
\else
	\medskip
	\section{Awards}
\cvlistitem{Fulbright Foreign Student Scholarship for Master's program, 2009-2011.}
\cvlistitem{2\textsuperscript{nd} place winner team at IEEE/ACM Supercomputer Conference 2009 (SC '09) High Performance Computing Contest.}
\cvlistitem{Google Summer of Code, 2008.}
\cvlistitem{Palestinian Prime Minister's special award for undergraduate achievements, 2008.}
	\medskip
    \section{Presentations and invited talks}
 
\cventry{}{ Network-wide Configuration Synthesis}{}{}{}{\begin{itemize}
\item Computer Aided Verification (CAV'17), \href{https://youtu.be/dUbFWtHLTCI}{https://youtu.be/dUbFWtHLTCI}, July 2017, Heidelberg Germany.
\item Workshop on Network Verification \href{http://network-programming.org/wnv/}{http://network-programming.org/wnv/}. April 2017. Palo Alto, CA, USA.
\end{itemize}
}
 
\cventry{}{BigBug: Practical Concurrency Analysis for SDN }{}{}{}{\begin{itemize}
\item ACM SIGPLAN conference on Programming Language Design and Implementation (PLDI'16), talk available at \href{https://youtu.be/QzYBOc3G9FU}{https://youtu.be/QzYBOc3G9FU}. June 2016. Santa Barbara, CA, USA.
\end{itemize}
}

\cventry{}{SDNRacer: Concurrency Analysis for Software-Defined Networks}{}{}{}{\begin{itemize}
\item ACM Symposium on SDN Research (SOSR). April 2017. Santa Clara, CA, USA.
\end{itemize}
}

\cventry{}{UNIS: Design and Implementation of a Unified Network Information Service}{}{}{}{\begin{itemize}
\item IEEE Conference on Services Computing (IEEE SCC). July 2013. Santa Clara, CA USA.
\end{itemize}
}
	\medskip
    \section{Publications}
\medskip
\nociteconference{*}
\bibliographystyleconference{IEEEtran}
\bibliographyconference{IEEEabrv,conference}                   % 'publications' is the name of a BibTeX file

\medskip
\medskip
\nociteworkshop{*}
\bibliographystyleworkshop{IEEEtran}
\bibliographyworkshop{workshop}                   % 'publications' is the name of a BibTeX file

\medskip
\medskip
\nocitetrs{*}
\bibliographystyletrs{IEEEtran}
\bibliographytrs{tr}                   % 'publications' is the name of a BibTeX file
\medskip
\nocitedemos{*}
\bibliographystyledemos{IEEEtran}
\bibliographydemos{demos}                   % 'publications' is the name of a BibTeX file

	\medskip
	%\section{Recent Research Projects}

\cvlistitem {\textbf{SDNv2}: We designed new a network architecture that changes the innovation model in the network.  The main architectural changes we propose are (1) using software based at the network edge (2) including middleboxes as a fundamental component in the architecture, as opposed to the current architectures that ignore their existence (3) extending network virtualization to higher-level services to open the network for third-party services and network function virtualization (NFV).}

%The Software-Defined Networking (SDN) paradigm, as canonically exemplified by the universal adoption of OpenFlow in hardware switches controlled by logically centralized controllers, was based on several implicit assumptions about networks. In this project we revisit those assumptions and find them wanting. We then propose a revised approach, which we call SDNv2 that involves (i) extending network virtualization to higher-level services, and (ii) implementing these and other functionalities on a more modular and flexible SDN architecture.}

\cvlistitem {\textbf{STS}: Industry providers for Software-Defined Networks (SDN) controller built large testing infrastructures to test their controllers. These testing infrastructures stress the controllers for long period of time, to simulate real network operating scenarios, until it find a bug in the controller. We have been told by some of these vendors that the input event traces that leads to that bug is large and it's very hard to troubleshoot a bug. Our work focuses on how to minimize the input event traces to make it reasonable enough for a developer to troubleshoot the discovered bug. We apply our technique to five open source SDN control platforms --Foodlight, NOX, POX, Pyretic, ONOS-- and illustrate how the minimal causal sequences our system found aided the troubleshooting process.\\ ONOS quality assurance team uses parts of \emph{STS} for integration and continuous operation test suits.}

%Software bugs are inevitable in software-defined networking control software, and troubleshooting is a tedious, time-consuming task. We worked on developing techniques for automatically identifying a minimal sequence of inputs responsible for triggering a given bug in control software, without making assumptions about the language or instrumentation of the software under test. We apply our technique to five open source SDN control platforms --Foodlight, NOX, POX, Pyretic, ONOS-- and illustrate how the minimal causal sequences our system found aided the troubleshooting process.\\
% ONOS quality assurance team uses parts of \emph{STS} for integration and continuous operation test suits.}

\cvlistitem { \textbf{GEMINI}: Develop and deploy instrumentation and measurement framework, apable of supporting the needs of both GENI experimenters and GENI infrastructure operators. It uses the perfSONAR system as its basis, and includes additional capabilities being developed by key GENI I\&M projects. It collects and manages both substrate metrics as well as slice-specific measurements. It introduces a GENI Global I\&M Information Service so that GEMINI components can locate one another, locate measurement sources and data, and correlate measurement data to physical and virtual network topology. It includes access control for instrumentation infrastructure, measurements, and measurement data based on GENI policy and authorization mechanisms.\\ \emph{GEMINI} tools are currently used to instrument and measure GENI experiments: \href{http://groups.geni.net/geni/wiki/GEMINI/Tutorial}{http://groups.geni.net/geni/wiki/GEMINI/Tutorial}.\\ \emph{GEMINI} APIs and data models are used by the GENI network operators to to expose physcial infrastructure information to the users \href{http://groups.geni.net/geni/wiki/OperationalMonitoring/DataSchema}{http://groups.geni.net/geni/wiki/OperationalMonitoring/DataSchema}. }
	%\medskip
    \section{Professional Experience}


% -------------- ETH--------------------
\cventry{June '15 -- Present}{Research Assistant}{ETH Z\"urich}{Switzerland}{}{\begin{itemize}
  \item Developed systems to automatically synthesize router configurations from high-level requirements.  \httplink[http://netcomplete.ethz.ch]{http://netcomplete.ethz.ch}, \httplink[http://synet.ethz.ch]{http://synet.ethz.ch}. Published in \cite{netcomplete, synet-cav, synet-arxiv}.
  \item Developed a system to detect concurrency violations in production-grade controllers of Software-Defined Networks (SDN).  \httplink[http://sdnracer.ethz.ch]{http://sdnracer.ethz.ch}. Published in \cite{bigbug, sdnracer2, sdnracer}.
  \item \emph{Technologies used:} Cisco IOS, BGP, OSPF, Python, SDN, Z3 SMT Solver, Git.
\end{itemize}}


% -------------- Facebook--------------------
\cventry{June -- Sept. '18}{Software Engineer Intern}{Facebook}{Menlo Park, CA, USA}{}{\begin{itemize}
  \item Started a practical network verification initiative.
  \item Built a prototype system to verify the correctness of the forwarding state of Facebook's network.
  \item \emph{Technologies used:} Python, Thrift, Mercurial.
\end{itemize}}


% -------------- IU 2nd --------------------
\cventry{Jan -- May '15}{Research Associate}{Indiana University}{Bloomington, IN, USA}{}{\begin{itemize}
\item Measured garbage collector and data serialization overhead for unstructured data.
\item Developed efficient method for representing unstructured data in Haskell's runtime system. Published in \cite{cnf}.
\item \emph{Technologies used:} Java, Git.
\end{itemize}}

% -------------- ICSI --------------------
\cventry{July '13 -- Nov. '14}{Research Scientist}{International Computer Science Institute (ICSI)}{Berkeley, CA, USA}{}{\begin{itemize}
\item Worked in Prof. Scott Shenker's group on designing and building next-generation SDN architecture (SDNv2).
\item Integrated a research system for quality assurance to run in a production environement with ONOS; a carrier-grade SDN open-source network operating system. \httplink[http://onosproject.org/]{http://onosproject.org}. Published in \cite{scott2014trooubleshooting}.
%\item Helped ONOS Quality Assurance team adopt parts of STS in their testing infrastructure.
\item \emph{Technologies used:} Python, Java, Continuous Integration QA, Git.
\end{itemize}}

% -------------- ESnet --------------------
\cventry{May -- July '13}{Summer Student}{Lawrence Berkeley National Lab.}{Berkeley, CA, USA}{}{\begin{itemize}
\item Worked with the Energy Sciences Network (ESnet) team.
\item Developed a scalable multi-domain Network Topology Service for dynamic multi-domain network circuits' setup.
\item \emph{Technologies used:} Java, Jetty, Atmoshper Framework, REST API design, MongoDB.
\end{itemize}}

% -------------- IU --------------------
\cventry{Aug. '11 -- July '13}{Research Associate}{Indiana University}{Bloomington, IN, USA}{}{\begin{itemize}
\item Developed a RESTful Unified Network Information Services (UNIS) to represent topologies for large-scale multi-domain networks. Published in \cite{ahassany2013unis}.
\item Developed instrumentation and monitoring APIs and tools for GENI \httplink[https://www.geni.net]{https://www.geni.net} experiments and physical infrastructure; \httplink[https://github.com/periscope-ps/unis]{github.com/periscope-ps/unis}.
\item \emph{Technologies used:} Java, Python, Tornado, REST API design, MongoDB, Django, Git.
\end{itemize}}

% --------------- SC -- maybe ignore it --------

% -------------- LBL --------------------
\cventry{August '10}{Summer Student}{Lawrence Berkeley National Lab.}{Berkeley, CA, USA}{}{\begin{itemize}
\item Worked at the Center for Enabling Distributed Petascale Computing (CEDPS) project.
\item Developed a system to collect, represent and analyze large scale  monitoring data for high-speed data transfers in DOE networks.
\item \emph{Technologies used:} Python, Tornado, JavaScript, HTML, JQuery, MongoDB, Django.
\end{itemize}}


% -------------- MoG --------------------
\cventry{Feb. '09 -- July '09}{Software Engineer}{Municipality of Gaza}{Gaza, Palestine}{}{\begin{itemize}
\item Designed a system to extract information, categorize, and archive old building permits from the late 1800s.
\item Designed a new business process and the required software for issuing new building permits in Gaza city.
\item Lead a team of 50 civil engineers and 20 data entry persons to implement the project.
\item \emph{Technologies used:} Oracle DB, Visual Basic~.NET, Delphi.
\end{itemize}}


% -------------- PNINA --------------------
\cventry{Dec. '08 -- Mar. '09}{Independent Consultant}{Palestinian National Internet Naming Authority (PNINA)}{Gaza, Palestine}{}{\begin{itemize}
\item Software quality assurance of in-house built system for registering and managing all \emph{.ps} domain names; PNINA is the country code top-level domain (ccTLD) for Palestine.
\item Consulting on deployment and integration for new domain registration system.
\item \emph{Technologies used:} PHP, Perl, PostgreSQL, MySQL, BIND, Apache.
\end{itemize}}

% -------------- Google Summer of Code --------------------
\cventry{June -- Sept '08}{Student Developer}{Google Inc. \& Internet2}{}{Google Summer of Code 2008}{\begin{itemize}
\item This project is sponsored by Google and administrated by Internet2.
\item Designed and developed open source web based configuration tools for perfSONAR-PS.
\item \emph{Technologies used:} Perl, JavaScript, Apache, HTML, CSS, SVN.
\end{itemize}}

% -------------- AfkarIT --------------------
\cventry{Sep. '07 -- Mar. '08}{Software Engineer}{AfkarIT}{}{Gaza, Palestine}{\begin{itemize}
\item Developed operating system level virtual machines monitoring system for VMWare ESX Infrastructure.
\item \emph{Technologies used:} Python, C\#.
\end{itemize}}


% -------------- Ard El-Insan --------------------
\cventry{Sep. '04 -- Oct. '05}{Contractor Software Engineer}{Ard El-Insan}{}{Gaza, Palestine}{\begin{itemize}
\item Developed patients follow-up management system in four clinics for a project sponsored by European Commission's Humanitarian Aid Office (ECHO).
\item \emph{Technologies used:} Visual Basic \mbox{.NET}, Microsoft SQL Server.
\end{itemize}}

	
	\medskip
	\section{Activities and Services}
\cvlistitem{Reviewer: IEEE/ACM Transactions on Networking.}
\cvlistitem{Attended Par Lab Boot Camp -- short course on parallel programming, UC Berkeley 2010.}
\cvlistitem{SCinet student volunteer, Supercomputing Conference 2010, 2011, 2012.}
\cvlistitem{President of Graduate Students Association for Computer and Information Science Department at University of Delaware, (Fall 2010 -- Spring 2011)}
\cvlistitem{Attended From Lab to Market, Fulbright Enrichment Seminar, Austin TX, June 1\textsuperscript{st}-5\textsuperscript{th} 2009.}
	\medskip
	\section{Supervised Students}
% Swisscom
\cvlistitem{Germain Zouein, Master Thesis, ``Graph-based Management and Visualization of Network Infrastructure,'' Swisscom Digital Lab/EPFL, 2021.}
\cvlistitem{Andrej Janchevski, Master Thesis, ``Graph Embedding Methods for Graph Completion,'' Swisscom Digital Lab/EPFL, 2021.}
\cvlistitem{Vincent Coriou, Master Thesis, ``Knowledge Graph Reasoning for Infrastructure Elements,'' Swisscom Digital Lab/EPFL, 2021.}
\cvlistitem{Nicolas Zimmermann, Master Thesis, ``Link Prediction on Knowledge Graphs for Infrastructure Elements,'' Swisscom Digital Lab/EPFL, 2021.}
% ETH
\cvlistitem{Alexander Hedges, Semester Thesis, ``Grigori: Continuous Integration Testing of Synthesize Router Configurations,'' ETH Z\"urich, 2018.}
\cvlistitem{Christelle Gloor, Semester Thesis, ``Chronos: Finding the Configurations Recipe for Fast Convergence,'' ETH Z\"urich, 2017.}
\cvlistitem{Roman May, Master Thesis, ``BigBug: Practical Concurrenty Analaysis for SDN,'' ETH Z\"urich, 2016.}


	
\fi
\end{document}
