\section{Recent Research Projects}

\cvlistitem {\textbf{SDNv2}: We designed new a network architecture that changes the innovation model in the network.  The main architectural changes we propose are (1) using software based at the network edge (2) including middleboxes as a fundamental component in the architecture, as opposed to the current architectures that ignore their existence (3) extending network virtualization to higher-level services to open the network for third-party services and network function virtualization (NFV).}

%The Software-Defined Networking (SDN) paradigm, as canonically exemplified by the universal adoption of OpenFlow in hardware switches controlled by logically centralized controllers, was based on several implicit assumptions about networks. In this project we revisit those assumptions and find them wanting. We then propose a revised approach, which we call SDNv2 that involves (i) extending network virtualization to higher-level services, and (ii) implementing these and other functionalities on a more modular and flexible SDN architecture.}

\cvlistitem {\textbf{STS}: Industry providers for Software-Defined Networks (SDN) controller built large testing infrastructures to test their controllers. These testing infrastructures stress the controllers for long period of time, to simulate real network operating scenarios, until it find a bug in the controller. We have been told by some of these vendors that the input event traces that leads to that bug is large and it's very hard to troubleshoot a bug. Our work focuses on how to minimize the input event traces to make it reasonable enough for a developer to troubleshoot the discovered bug. We apply our technique to five open source SDN control platforms --Foodlight, NOX, POX, Pyretic, ONOS-- and illustrate how the minimal causal sequences our system found aided the troubleshooting process.\\ ONOS quality assurance team uses parts of \emph{STS} for integration and continuous operation test suits.}

%Software bugs are inevitable in software-defined networking control software, and troubleshooting is a tedious, time-consuming task. We worked on developing techniques for automatically identifying a minimal sequence of inputs responsible for triggering a given bug in control software, without making assumptions about the language or instrumentation of the software under test. We apply our technique to five open source SDN control platforms --Foodlight, NOX, POX, Pyretic, ONOS-- and illustrate how the minimal causal sequences our system found aided the troubleshooting process.\\
% ONOS quality assurance team uses parts of \emph{STS} for integration and continuous operation test suits.}

\cvlistitem { \textbf{GEMINI}: Develop and deploy instrumentation and measurement framework, apable of supporting the needs of both GENI experimenters and GENI infrastructure operators. It uses the perfSONAR system as its basis, and includes additional capabilities being developed by key GENI I\&M projects. It collects and manages both substrate metrics as well as slice-specific measurements. It introduces a GENI Global I\&M Information Service so that GEMINI components can locate one another, locate measurement sources and data, and correlate measurement data to physical and virtual network topology. It includes access control for instrumentation infrastructure, measurements, and measurement data based on GENI policy and authorization mechanisms.\\ \emph{GEMINI} tools are currently used to instrument and measure GENI experiments: \href{http://groups.geni.net/geni/wiki/GEMINI/Tutorial}{http://groups.geni.net/geni/wiki/GEMINI/Tutorial}.\\ \emph{GEMINI} APIs and data models are used by the GENI network operators to to expose physcial infrastructure information to the users \href{http://groups.geni.net/geni/wiki/OperationalMonitoring/DataSchema}{http://groups.geni.net/geni/wiki/OperationalMonitoring/DataSchema}. }